\documentclass[UTF8]{ctexart}
\usepackage{amsmath}
\usepackage{amssymb}
\usepackage{graphics}

\begin{document}

\paragraph*{椅子问题}

\subparagraph*{假设:} 
(1) 椅子是长方形的,四条腿一样长,四条腿的连线同样成长方形。(2)地面是数学上的连续曲面。

\subparagraph*{解:}
设$A,B,C,D$为长方形椅子的四个脚。其中$A,B$两脚和$C,D$两脚与地面的距离之和分别为$f(\theta), g(\theta)$。其中$\theta$为椅面绕其法向量旋转的角度。

椅子在任意位置,总有三只脚可以着地,故而对于任意$\theta$总有$f(\theta),g(\theta)$中至少一个为零。不妨假设$g(0)=0,f(0)>0$,且$f(\theta),g(\theta)$均为连续函数。

现在需要证明:存在$\theta_0$,使得$f(\theta_0)=g(\theta_0)=0$,此时椅子即放平稳。
作辅助函数:$$h(\theta)=f(\theta)-g(\theta)$$
现将椅子旋转$180^\circ$,则对边的椅子腿互换,有:
$$f(\pi)=0,g(\pi)>0$$
则$h(x)$也是连续函数,且$h(0)>0,h(\pi)<0$,由连续函数的介值定理可知,比存在$\theta_0 \in (0,\pi)$使得$h(\theta_0)=0$, 则有$g(\theta_0)=f(\theta_0)=0$,\textbf{证毕}。


\paragraph*{单摆问题}

\subparagraph*{假设:}(1)单摆受到的阻力和速度成正比,比例系数为$k$。(2)单摆的周期$t$,质量为$m$,重力加速度$g$,摆长$l$,受到的阻力为$F$,摆的速度为$v$,摆动幅角$\theta$。

\subparagraph*{解:}
上述物理量(按出现的先后顺序)的量纲分别为$MT^{-1},T,M,LT^{-2}$ $MLT^{-2},LT^{-1}$,
($\theta$为无量纲量)构造无量纲量:
$$
\sigma = k^{p_1}t^{p_2}m^{p_3}g^{p_4}l^{p_5}F^{p_6}v^{p_7}\theta^{p_8}
$$
则有其量纲:
$$
M^{p_1+p_3+p_6}L^{p_4+p_5+p_6+p_7}T^{-p_1+p_2-2p_4-2p_6-p_7}
$$
故而
$$
\begin{cases}
    p_1+p_3+p_6=0\\
    p_4+p_5+p_6+p_7=0\\
    -p_1+p_2-2p_4-2p_6-p_7=0\\
\end{cases}
$$
其中$p_8$可以任取。
原方程组可以写成:
$$
\begin{pmatrix}
    1&0&1&0&0&1&0\\
    0&0&0&1&1&1&1\\
    -1&1&0&-2&0&-2&-1
\end{pmatrix}
\begin{pmatrix}
    p_1\\
    p_2\\
    p_3\\
    p_4\\
    p_5\\
    p_6\\
    p_7\\
\end{pmatrix}
=0
$$
解得原方程组有四个基础解向量,加上$\theta$的次数后一共有五个,如下:
$$
\xi_1 = (-1,-1,1,0,0,0,0,0)^T
$$
$$
\xi_2 = (0,-2,0,-1,1,0,0,0)^T
$$
$$\xi_3 = (-1,-1,0,-1,0,1,0,0)^T$$
$$\xi_4 = (0,-1,0,-1,0,0,1,0)^T$$
$$\xi_5 = (0,0,0,0,0,0,0,1)^T$$
故而有:
$$\sigma_1 =\frac{m}{kt}$$
$$\sigma_2 = \frac{l}{gt^2}$$
$$\sigma_3 =\frac{F}{kgt}$$
$$\sigma_4 = \frac{v}{gt}$$
$$\sigma_5 = \theta$$
根据 Buckingham $\Pi $定理,有$Q(\sigma_1,\cdots,\sigma_5)=0$与$q(k,m,t,l,g,v,F,\theta)=0$等价
故而$Q(\sigma_1,\cdots,\sigma_5)=0$即为$t$的表达式。














\end{document}