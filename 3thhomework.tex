\documentclass[UTF8]{ctexart}
\usepackage{amsmath}
\usepackage{amssymb}
\usepackage{ulem}
\usepackage{graphicx}
\begin{document}

\paragraph*{Merchants crossing the river}

\subparagraph*{Basic hypothesis:} 

(1)There are 3 merchants, each with a servant, who all need to cross the river (from river bank $A$ to river bank $B$).(2)The boat can only take 3 people. (3) On either side of the river, Merchants will be killed by  servants as long as there are more servants than merchants. 

\subparagraph*{Solution:}
Let $s$ be the number of servant on river bank $A$, and $m$ be the number of merchants on river bank $A$, and $(s,m)$ is a status. 

According to the hypothesis, there are 16 statuses, but only the statuses without a line through the center are allowed.
\begin{center}
    \begin{tabular}{ c c c c}
     $(3,3)$ & \sout{$(3,2)$} & \sout{$(3,1)$} & $(3,0)$\\ 
     $(2,3)$ & $(2,2)$ & \sout{$(2,1)$} & $(2,0)$\\  
     $(1,3)$ & \sout{$(1,2)$} & $(1,1)$ & $(1,0)$\\
     $(0,3)$ & \sout{$(0,2)$} & \sout{$(0,1)$} & $(0,0)$
    \end{tabular}
\end{center}
We now put these statuses as coordinates and draw them out on a plat surface:
\begin{figure}[htbp]
    \includegraphics[width=\textwidth]{/Users/zitingyuan/Desktop/mathematic modelling/usual homework/Script.png}
\end{figure}

\noindent and we can operate obeying the following rules.\\
(a) for all operation of the $m$th ($m$ is odd), go leftward or downward for no more than 3 boxes in total\\
(b)for all operation of the $n$th ($n$ is even), go rightward or upward for no more than 3 boxes in total\\

Then we can plan the following route:
\begin{figure}[htbp]
    \includegraphics[width=\textwidth]{Script2.png}
\end{figure}

And thus we got the answer!
















\end{document}