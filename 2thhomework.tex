\documentclass[UTF8]{ctexart}
\usepackage{amsmath}
\usepackage{amssymb}
\usepackage{graphicx}
\newcommand{\df}{\mathrm{~d}}

\begin{document}

\paragraph{淋雨问题}

\subparagraph{假设1:}(1)人面前的雨滴在空间中是稠密、连续的,即空间中的每一点都有雨滴,不存在“断点”。(2)将雨滴中的人体近似成一个圆柱体,高$h_0$,半径$r_0$,行走速度$\vec{v}$。

\subparagraph{解:}由于这场雨近似稠密,故而可以把雨看作一个流场,即可描述该流场中每一点的速度:
$$\mathbf{v}(x,y,z)=X(x,y,z)\vec{i}+Y(x,y,z)\vec{j}+Z(x,y,z)\vec{k}$$
三个向量分别描述雨滴沿$x$,$y$,$z$三轴的速度。

然后可以给出人体的的曲面$S$(将雨滴中的人体近似成一个圆柱体):首先是人的头部
$$
S_1 :
\begin{cases}
    x^2+y^2=r_0\\
    z=h_0
\end{cases}
$$
然后是身体:
$$
S_2:
x^2+y^2=r_0 ~~ (0 \leqslant z < h_0)
$$
脚下被挡住,可以近似看作不会淋到雨,故而不用给出方程,即$S=S_1 \cup S_2$。假设$S$上每一点的法向量是$\vec{n}(x,y,z)$。

现在求出雨场相对人体的速度,就是$\mathbf{v}-\vec{v}$。故而其单位时间内打到人体上的雨水体积即:
$$
\left | \iint_{S}(\mathbf{v}-\vec{v}) \cdot \vec{n} \df S \right | = \left |\iint_S X-v_x \df y \df z + \iint_S Y-v_y \df x \df z + \iint_S Z-v_z \df x \df y \right |
$$
这是一个较为复杂的带参量的二重积分,并不方便计算,但是等式左端的积分式与高中电磁学计算磁通量的相似性提示我们:它可以简化为有关人体曲面对于流场的投影面的积分。即如下:
$$
\left | \iint_{S}(\mathbf{v}-\vec{v}) \cdot \vec{n} \df S \right | = \left | \iint_{S_1}(\mathbf{v}-\vec{v}) \cdot \vec{n_1} \df S_1 \right |+ \left |\iint_{S_2}(\mathbf{v}-\vec{v}) \cdot \vec{n_2} \df S_2  \right |
$$
(实际上由于雨场都是“穿过”曲面外侧的,两个部分的积分符号不变,所以能直接分成两个绝对值)

我们不妨将某一区域内的雨滴近似地认为运动方向相同(生活经验), 而且人通常不会左脚踩右脚上天($v_z=0$)即:
$$\mathbf{v}-\vec{v}=(a-v_x,b-v_y,c)$$
$S_2$曲面的法向量为$\vec{n_2}=(0,0,1)$,关于$S_2$的积分式可以简化为
$$
A_{S_2 }\cdot |\cos <\mathbf{v}-\vec{v},\vec{n_2 }>| \cdot |\mathbf{v}-\vec{v}|
$$
$$
=
\pi r_0^2 \cdot |c|
$$
对$S_2$表面来讲,单位时间内淋雨的量和人的跑动方向以及速度没关系,人的速度越快,到达避雨的地方时间越短,淋雨淋得越少。

对于$S_1$的讨论也一样。由于$S_1$关于$z$轴具有旋转不变性,于是我们不妨假设:雨场在地面的投影正交于这个圆柱的一个切面如图:
\begin{figure}[htbp]
    \includegraphics[width=\textwidth]{/Users/zitingyuan/Desktop/mathematic modelling/usual homework/WechatIMG1145.jpg}
\end{figure}

中间虚线y轴即是前文所言切面。则此时

$$
\iint_{S_1}(\mathbf{v}-\vec{v}) \cdot \vec{n_1} \df S_1 = 2r_0h_0 \cdot \cos<\mathbf{v}-\vec{v},\vec{n_1}> |\mathbf{v}-\vec{v}| = 2r_0h_0 \cdot (a-v_x)
$$
单位时间内人体柱面部分的淋雨量与人行走的速度有关。

假设人行走的距离为$l$。有:
$$
V = t \cdot \iint_{S_1}(\mathbf{v}-\vec{v}) \cdot \vec{n_1} \df S_1 = \frac{l}{\sqrt{v_x^2+v_y^2}}(a-v_x)
$$
依然有走得越快淋的越少。

故而,我们可以得出结论,走路越快,淋雨越少。以后下雨走快点到目的地就不用打伞了。
















\end{document}